\let\negmedspace\undefined
\let\negthickspace\undefined
%\RequirePackage{amsmath}
\documentclass[journal,12pt,twocolumn]{IEEEtran}
\usepackage{gensymb}
\usepackage{polynom}
\usepackage{amssymb}
\usepackage[cmex10]{amsmath}
\usepackage{amsthm}
\usepackage{stfloats}
\usepackage{bm}
\usepackage{enumitem}
\usepackage{mathtools}
  %\usepackage{tikz}
% \usepackage{circuitikz}
% \usepackage{verbatim}
%\usepackage{tfrupee}
  %\usepackage[breaklinks=true]{hyperref}
%\usepackage{stmaryrd}
%\usepackage{tkz-euclide} % loads  TikZ and tkz-base
%\usetkzobj{all}
\usepackage{listings}
    \usepackage{color}                                            
    \usepackage{array}                                            
    \usepackage{longtable}                                        
    \usepackage{calc}                                             
    \usepackage{multirow}                                         
    \usepackage{hhline}                                           
    \usepackage{ifthen}                                           
  %optionally (for landscape tables embedded in another document): 
    \usepackage{lscape}     
% \usepackage{multicol}
% \usepackage{chngcntr}
%\usepackage{enumerate}
%\usepackage{wasysym}
%\newcounter{MYtempeqncnt}
\DeclareMathOperator*{\Res}{Res}
\DeclareMathOperator*{\equals}{=}
%\renewcommand{\baselinestretch}{2}
\renewcommand\thesection{\arabic{section}}
\renewcommand\thesubsection{\thesection.\arabic{subsection}}
\renewcommand\thesubsubsection{\thesubsection.\arabic{subsubsection}}
\renewcommand\thesectiondis{\arabic{section}}
\renewcommand\thesubsectiondis{\thesectiondis.\arabic{subsection}}
\renewcommand\thesubsubsectiondis{\thesubsectiondis.\arabic{subsubsection}}
% correct bad hyphenation here
\hyphenation{op-tical net-works semi-conduc-tor}
\def\inputGnumericTable{}                                 
\lstset{
%language=C,
frame=single, 
breaklines=true,
columns=fullflexible
}
%
\begin{document}
%
\newtheorem{theorem}{Theorem}[section]
\newtheorem{problem}{Problem}
\newtheorem{proposition}{Proposition}[section]
\newtheorem{lemma}{Lemma}[section]
\newtheorem{corollary}[theorem]{Corollary}
\newtheorem{example}{Example}[section]
\newtheorem{definition}[problem]{Definition}
%\newtheorem{thm}{Theorem}[section] 
%\newtheorem{defn}[thm]{Definition}
%\newtheorem{algorithm}{Algorithm}[section]
%\newtheorem{cor}{Corollary}
\newcommand{\BEQA}{\begin{eqnarray}}
\newcommand{\EEQA}{\end{eqnarray}}
\newcommand{\define}{\stackrel{\triangle}{=}}
\newcommand*\circled[1]{\tikz[baseline=(char.base)]{
    \node[shape=circle,draw,inner sep=2pt] (char) {#1};}}
\bibliographystyle{IEEEtran}
%\bibliographystyle{ieeetr}
%
\providecommand{\mbf}{\mathbf}
\providecommand{\pr}[1]{\ensuremath{\Pr\left(#1\right)}}
\providecommand{\qfunc}[1]{\ensuremath{Q\left(#1\right)}}
\providecommand{\sbrak}[1]{\ensuremath{{}\left[#1\right]}}      % []
\providecommand{\lsbrak}[1]{\ensuremath{{}\left[#1\right.}}
\providecommand{\rsbrak}[1]{\ensuremath{{}\left.#1\right]}}
\providecommand{\brak}[1]{\ensuremath{\left(#1\right)}}         % ()
\providecommand{\lbrak}[1]{\ensuremath{\left(#1\right.}}
\providecommand{\rbrak}[1]{\ensuremath{\left.#1\right)}}
\providecommand{\cbrak}[1]{\ensuremath{\left\{#1\right\}}}      % {}
\providecommand{\lcbrak}[1]{\ensuremath{\left\{#1\right.}}
\providecommand{\rcbrak}[1]{\ensuremath{\left.#1\right\}}}
\theoremstyle{remark}
\newtheorem{rem}{Remark}
\newcommand{\sgn}{\mathop{\mathrm{sgn}}}
\providecommand{\abs}[1]{\ensuremath{\left\vert#1\right\vert}}
\providecommand{\res}[1]{\Res\displaylimits_{#1}} 
\providecommand{\norm}[1]{\ensuremath{\left\lVert#1\right\rVert}}
%\providecommand{\norm}[1]{\lVert#1\rVert}
\providecommand{\mtx}[1]{\mathbf{#1}}
\providecommand{\mean}[1]{\ensuremath{E\left[ #1 \right]}}
\providecommand{\fourier}{\overset{\mathcal{F}}{ \rightleftharpoons}}
%\providecommand{\hilbert}{\overset{\mathcal{H}}{ \rightleftharpoons}}
\providecommand{\system}{\overset{\mathcal{H}}{ \longleftrightarrow}}
	%\newcommand{\solution}[2]{\textbf{Solution:}{#1}}
\newcommand{\solution}{\noindent \textbf{Solution: }}
\newcommand{\cosec}{\,\text{cosec}\,}
\providecommand{\dec}[2]{\ensuremath{\overset{#1}{\underset{#2}{\gtrless}}}}
\newcommand{\myvec}[1]{\ensuremath{\begin{pmatrix}#1\end{pmatrix}}}
\newcommand{\mydet}[1]{\ensuremath{\begin{vmatrix}#1\end{vmatrix}}}
\newcommand*{\permcomb}[4][0mu]{{{}^{#3}\mkern#1#2_{#4}}}
\newcommand*{\perm}[1][-3mu]{\permcomb[#1]{P}}
\newcommand*{\comb}[1][-1mu]{\permcomb[#1]{C}}
%
%not used because document is short:
%\numberwithin{equation}{section}
%\numberwithin{figure}{section}
%\numberwithin{table}{section}
%\numberwithin{equation}{section}
%\numberwithin{problem}{section}
%\numberwithin{definition}{section}
\makeatletter
\@addtoreset{figure}{problem}
\makeatother

\let\StandardTheFigure\thefigure
\let\vec\mathbf
%\renewcommand{\thefigure}{\theproblem.\arabic{figure}}
    %\renewcommand{\thefigure}{\theproblem}
%\setlist[enumerate,1]{before=\renewcommand\theequation{\theenumi.\arabic{equation}}
%\counterwithin{equation}{enumi}
%\renewcommand{\theequation}{\arabic{subsection}.\arabic{equation}}

\def\putbox#1#2#3{\makebox[0in][l]{\makebox[#1][l]{}\raisebox{\baselineskip}[0in][0in]{\raisebox{#2}[0in][0in]{#3}}}}
     \def\rightbox#1{\makebox[0in][r]{#1}}
     \def\centbox#1{\makebox[0in]{#1}}
     \def\topbox#1{\raisebox{-\baselineskip}[0in][0in]{#1}}
     \def\midbox#1{\raisebox{-0.5\baselineskip}[0in][0in]{#1}}
\vspace{3cm}

\title{Assignment 3 \\ 12.13.1.15}
\author{K.SaiTeja \\ AI22BTECH11014}	

\maketitle
%\newpage
%\tableofcontents
%\bigskip
%\renewcommand{\thefigure}{\theenumi}
%\renewcommand{\thetable}{\theenumi}
%\renewcommand{\theequation}{\theenumi}
\textbf{Question 15:}
Consider the experiment of throwing a die. If a multiple of 3 comes up, throw the die again and if any other number comes, toss a coin. Find the conditional probability of the event \lq the coin shows a tail\rq, given that \lq at least one die shows a 3\rq.\\
\solution
Let us define some random variables.
\begin{table}[ht!]
    \centering
	\input{tables/table1.tex}
    \caption{Random variables}
    \label{table:RandomVariables}	
\end{table}

When the die rolls a multiple of 3, recursion is generated. For $k \in \cbrak{1,2,3,4,5,6}$,
\begin{align}
    \pr{X_{n+1}=k} &= \sum_{i \in \cbrak{3, 6}}\pr{X_1=i} \times \pr{X_n=k}
    \label{eq:recursionX}
\end{align}
\begin{align}
    \pr{X_{n+1}=k} &= \frac{2}{6} \times \pr{X_n=k}\\
    \implies \pr{X_{n}=k} &= \brak{\frac{1}{3}}^{n-1} \times \pr{X_1=k}
\end{align}
There is a recursion on the first occurrence of 3.
\begin{align}
    \pr{Y=n} &= \pr{X_1=6} \times \pr{Y= n-1} \\
    \implies \pr{Y=n} &= \brak{\frac{1}{6}}^{n-1} \times \pr{X_1=3} 
    \label{eq:recursionY}
\end{align}
For probability that 3 occurs at least once,
\begin{align}
    \sum_{n=1}^\infty  \pr{Y=n} &= 
     \sum_{n=1}^\infty \brak{\frac{1}{6}}^{n-1} \times \pr{X_1=3} \\
    \sum_{n=1}^\infty \pr{Y=n} &= 
     \frac{\frac{1}{6}}{1-\frac{1}{6}} \times \frac{1}{6} \\
   \sum_{n=1}^\infty \pr{Y=n} &= \frac{1}{5}
   \label{eq:occurrence3}
\end{align}
%%
Required conditional probability is,
\begin{align}
    &\pr{\brak{Z_1=1} | \sum_{n=1}^\infty \pr{Y=n}} \\
    &=\frac{ \sum_{n=1}^\infty \pr{\brak{Y=n}\brak{Z_1 = 1}} } {\sum_{n=1}^\infty \pr{Y=n}}
\end{align}
%%
Consider first occurrence of 3 on $n^{th}$ throw and m further throws.
\begin{align}
    &\sum_{n=1}^\infty \pr{\brak{Y=n}\brak{Z_1 = 1}} \\
    &=\sum_{n=1}^\infty \sum_{m=1}^\infty \pr{Y=n} 
      \sum_{i \in \cbrak{1,2,4,5}}\pr{X_m = i} \times \pr{Z_2 = 1} \\
    &=\sum_{n=1}^\infty \pr{Y=n} \sum_{m=1}^\infty 
    4 \times \brak{\frac{1}{3}}^{m-1} \brak{\frac{1}{6}} \times \pr{Z_2 = 1} \\
    &= \frac{1}{5} \times 4 \times \frac{3}{2} \times \frac{1}{6} \times \frac{1}{2}
\end{align}
\begin{align}
    \implies \sum_{n=1}^\infty \pr{Y=n} \pr{Z_1 = 1} = \frac{1}{10}
\end{align}
%%
\begin{align}
    \therefore \pr{\brak{Z_1=1} | \sum_{n=1}^\infty \pr{Y=n}}  &= \frac{1}{10} \div \frac{1}{5} \\
    &= \frac{1}{2}
\end{align}
\end{document}

